\documentclass[12pt,a4paper] {article}

\usepackage{amsmath}
\usepackage[utf8]{inputenc}
\usepackage{structuralanalysis}
\usepackage{3dstructuralanalysis}
\usepackage[left=15mm, right=15mm, top=20mm, bottom=20mm,]{geometry}
\usepackage{multicol}
\usepackage{graphicx}
\usepackage{fancyhdr}
\usepackage{wasysym}
\usepackage{textcomp}

\pagestyle{fancy}
\fancyhf{}
\rhead{VEM Alapjai II. Házi Feladat}
\lhead{Gazdag Sándor, GV1MYC}
\rfoot{\thepage}
\graphicspath{ {images/} }
\setcounter{MaxMatrixCols}{20}
\begin{document}
	

%adatok


\section{Adatok:}
{\large Alapadatok: }\par
\begin{align*}
a&=2.6\, [m] & b&=5\, [m]\\
d&=0.025\, [m] & \rho&=6000\, [kg/m^2]\\
m_0&=30\, [kg] & E&=190\, [GPa]\\
\end{align*}\par
{\large Számított: }\par
\begin{align*}
A=\bigg(\dfrac{d}{2}\bigg)^2\cdot \pi=4,9087\cdot 10^{-4} \, [m] \qquad \qquad I=\dfrac{d^4\cdot \pi}{64}=1,9175\cdot 10^{-8}\, [m]\\
\end{align*}\par


%ábra


\vspace{20mm}
\section{Ábra:}
{\large Méretarányos ábra kényszerekkel:} (\textit{Bár $m_0$ pontszerű, itt egy körrel ábrázoltam.})\par
\vspace*{10mm}
\begin{tikzpicture}

	\scaling{.20};
	\point{A}{0}{0};
	\point{B}{50}{0};
	\point{C}{76}{0};
	
	\point{D}{80}{0};
	\point{E}{0}{15};
	
	\point{F}{77}{4};
	
	\draw[ dashed, ->] (A)--(D);
	\draw[ dashed, ->] (A)--(E);
		
	\point{A1}{0}{0.125};
	\point{A2}{0}{-0.125};
	\point{B1}{50}{0.125};
	\point{B2}{50}{-0.125};
	\point{C1}{76}{0.125};
	\point{C2}{76}{-0.125};
	
	\draw (A1)--(A2);
	\draw (A1)--(C1);
	\draw (A2)--(C2);
	\draw (C1)--(C2);
	
	\draw[thick, inner color=white] (C) circle (4mm);
	
	\support{2}{B};
	\support{3}{A}[-90];
	
	\dimensioning{1}{A}{B}{-1.5}[$5~m$];
	\dimensioning{1}{A}{C}{-2.5}[$2.6~m$];
	\dimensioning{2}{C1}{C2}{+17}[$\quad \diameter \,0.025\,m$];
	
	\notation{1}{D}{$x$}[below]
	\notation{1}{E}{$y$}[right]
	\notation{1}{A}{$A$}[above right]
	\notation{1}{B}{$B$}[above]
	\notation{1}{C}{$C$}[below=0.4]	
	\notation{1}{F}{$m_0=30 kg$}[]

\end{tikzpicture}\par
\newpage


%1. feladat


\section{Sajátfrekvencia a.}
{\large Pontszerű tömeg elhanyagolásával, 2 elemmel a végeselem ábra:}\par
\vspace{15mm}
\begin{tikzpicture}
	\scaling{.20};
	\point{A}{0}{0};
	\point{B}{50}{0};
	\point{C}{76}{0};
	
	\beam{2}{A}{B};
	\beam{2}{B}{C};
	
	\support{2}{B};
	\support{3}{A}[-90];
	
	\hinge{1}{A};
	\hinge{1}{B};
	\hinge{1}{C};
	
	\notation{1}{A}{$(1)$}[above right];
	\notation{1}{B}{$(2)$}[above];
	\notation{1}{C}{$(3)$}[above];

	\notation{5}{A}{B}[$I, E, A, \rho $][.5][below];
	\notation{5}{B}{C}[$I, E, A, \rho $][.5][below];

	\notation{4}{A}{B}[1]
	\notation{4}{B}{C}[2] 
\end{tikzpicture}\par
\vspace{15mm}
\noindent
A globális merevségi mátrixot a segédlet alapján állítottam össze az egyes rudakra kiszámolt merevségi mátrixokból. A számítások elvégzéséhez Maxima programot használtam.\par
\vspace{5mm}
\noindent
Az alábbi lépésekből állt az algoritmusom: \par
\begin{enumerate}
	\item Adatok.
	\begin{itemize}
		\item Megadott adatok bevitele.
		\item Egyéb adatok számítása.
		\item Adatok vektorokba rendezése.
	\end{itemize}
	\item Az elemi merevségi mátrix függvény és az elemi tömegmátrix függvény megvalósítása egy rúdra. Ezek 4 darab kis 2x2-es, a csomópontkombinációkra vonatkozó mátrixra bonthatóak:
	\[
	K_e= \frac{I \cdot E}{L^3}  
	\begin{bmatrix}
	12&6L&-12&6L\\
	6L&4L^2&-6L&2L^2\\
	-12&-6L&12&-6L\\
	6L&2L^2&-6L&4L^2\\
	\end{bmatrix}
	=\begin{bmatrix}
	K_{11} &K_{12} \\
	K_{21} &K_{22} \\
	\end{bmatrix}
	\]
	%tömegmátrix
	\[
	M_e= \frac{\rho \cdot A \cdot L}{420}  
	\begin{bmatrix}
	156&22L&54&-13L\\
	22L&4L^2&13L&-3L^2\\
	54&13L&156&-22L\\
	-13L&-3L^2&-22L&4L^2\\
	\end{bmatrix}
	=\begin{bmatrix}
	M_{11} &M_{12} \\
	M_{21} &M_{22} \\
	\end{bmatrix}
	\]
	\item Globális merevségi és tömegmátrix összeállítása az elemi merevségi és tömegmátrixokból az alábbi módon:
	\[K_G=
	\begin{bmatrix}
	K_{11}^1 &K_{12}^1 &0  \\
	K_{21}^1 &K_{22}^1+K_{11}^2 &K_{12}^2  \\
	0 &K_{21}^2 &K_{22}^2 \\
	\end{bmatrix}
	\]
	%tömeg
	\[M_G=
	\begin{bmatrix}
	M_{11}^1 &M_{12}^1 &0  \\
	M_{21}^1 &M_{22}^1+M_{11}^2 &M_{12}^2  \\
	0 &M_{21}^2 &M_{22}^2 \\
	\end{bmatrix}
	\]  
	\item Mátrixok kondenzálása a szabad szabadsági fokok alapján. Lekötött szabadsági fokok:
	\[V_{1}=0 \qquad \Phi_{1}=0 \qquad V_{2}=0\]
	\newpage
	\item Sajátkörfrekvenciák számítása az alábbi képlet segítségével:
	\[det(K_G-\omega^2 \cdot M_G)=0\]
	Tehát \(det(M_G^{-1} \cdot K_G)\) sajátértékei lesznek $\omega^2$ értékei.
	\item Maxima segítségével az így kapott sajátkörfrekvenciákból számítottam a sajátfrekvenciákat:
	\[f_{1,2,3}^a=\dfrac{\omega_{1,2,3}^a}{2\cdot \pi}\]
	Amivel a sajátfrekvenciák:	
	\[f_1^a=1,6197\, [Hz] \qquad f_2^a=7,6607\, [Hz] \qquad f_3^a=30,4388\, [Hz]  \]
	Az így számított eredmények legalább 3 tizedesjegyik megegyeznek a SIKEREZ szoftverrel kapott eredményekkel. A SIKEREZ file-t a Maxima file mellé csatolom "sikerb.skz" néven.
\end{enumerate}\par


%2. feladat


\section{Sajátfrekvencia b.}
{\large Pontszerű tömeg elhanyagolásával, 3 elemmel a végeselem ábra:}\par
\vspace{15mm}
\begin{tikzpicture}
\scaling{.20};
\point{A}{0}{0};
\point{B1}{25}{0};
\point{B}{50}{0};
\point{C}{76}{0};

\beam{2}{A}{B1};
\beam{2}{B1}{B};
\beam{2}{B}{C};

\support{2}{B};
\support{3}{A}[-90];

\hinge{1}{A};
\hinge{1}{B1};
\hinge{1}{B};
\hinge{1}{C};

\notation{1}{A}{$(1)$}[above right];
\notation{1}{B1}{$(2)$}[above];
\notation{1}{B}{$(3)$}[above];
\notation{1}{C}{$(4)$}[above];

\notation{5}{A}{B1}[$I, E, A, \rho $][.5][below];
\notation{5}{B1}{B}[$I, E, A, \rho $][.5][below];
\notation{5}{B}{C}[$I, E, A, \rho $][.5][below];

\notation{4}{A}{B1}[1]
\notation{4}{B1}{B}[2]
\notation{4}{B}{C}[3] 
\end{tikzpicture}\par
\vspace{15mm}
\noindent
Az előző feladathoz hasonló algoritmuson haladtam végig, szintén Maxima programmal.\par
\vspace{5mm}
\noindent
Az alábbi lépésekből állt az algoritmusom: \par
\begin{enumerate}
	\item Adatok bevitele.
	\item Az elemi merevségi mátrix függvény és az elemi tömegmátrix függvény megvalósítása az előző feladathoz hasonlóan.
	\item Globális merevségi és tömegmátrix összeállítása az elemi merevségi és tömegmátrixokból az alábbi módon:
	\[K_G=
	\begin{bmatrix}
	K_{11}^1 &K_{12}^1 &0 &0 \\
	K_{21}^1 &K_{22}^1+K_{11}^2 &K_{12}^2 &0 \\
	0 &K_{21}^2 &K_{22}^2+K_{11}^3 &K_{12}^3 \\
	0 &0 &K_{21}^3 &K_{22}^3 \\
	\end{bmatrix}
	\] 
	%tömeg
	\[M_G=
	\begin{bmatrix}
	M_{11}^1 &M_{12}^1 &0 &0 \\
	M_{21}^1 &M_{22}^1+K_{11}^2 &M_{12}^2 &0 \\
	0 &M_{21}^2 &M_{22}^2+M_{11}^3 &M_{12}^3 \\
	0 &0 &M_{21}^3 &M_{22}^3 \\
	\end{bmatrix}
	\]  
	\item Mátrixok kondenzálása a szabad szabadsági fokok alapján. A lekötött szabadsági fokok, az előző feladathoz képest nem változtak:
	\[V_{1}=0 \qquad \Phi_{1}=0 \qquad V_{2}=0\]
	\newpage
	\item Sajátkörfrekvenciák számítása az alábbi képlet segítségével:
	\[det(K_G-\omega^2 \cdot M_G)=0\]
	Tehát \(det(M_G^{-1} \cdot K_G)\) sajátértékei lesznek $\omega^2$ értékei.
	\item Maxima segítségével az így kapott sajátkörfrekvenciákból számítottam a sajátfrekvenciákat:
	\[f_{1,2,3}^b=\dfrac{\omega_{1,2,3}^b}{2\cdot \pi}\]
	Amivel a sajátfrekvenciák:
	\[f_1^b=1,6129\, [Hz] \qquad f_2^b=4,5404\, [Hz] \qquad f_3^b=13,5619\, [Hz]  \]
	Az így számított eredmények legalább 3 tizedesjegyik megegyeznek a SIKEREZ szoftverrel kapott eredményekkel. A SIKEREZ file-t a Maxima file mellé csatolom "sikerc.skz" néven.
\end{enumerate}\par


%3. feladat


\section{Sajátfrekvencia c.}
{\large Pontszerű tömeggel, 3 elemmel a végeselem ábra:}\par
\vspace{15mm}
\begin{tikzpicture}
	\scaling{.20};
	\point{A}{0}{0};
	\point{B1}{25}{0};
	\point{B}{50}{0};
	\point{C}{76}{0};
	
	\beam{2}{A}{B1};
	\beam{2}{B1}{B};
	\beam{2}{B}{C};
	
	\support{2}{B};
	\support{3}{A}[-90];
	
	\hinge{1}{A};
	\hinge{1}{B1};
	\hinge{1}{B};
	\hinge{1}{C};
	
	\notation{1}{A}{$(1)$}[above right];
	\notation{1}{B1}{$(2)$}[above];
	\notation{1}{B}{$(3)$}[above];
	\notation{1}{C}{$(4)$}[above=0.4];
	\notation{1}{C}{$m_0$}[right=0.4];
		
	\notation{5}{A}{B1}[$I, E, A, \rho $][.5][below];
	\notation{5}{B1}{B}[$I, E, A, \rho $][.5][below];
	\notation{5}{B}{C}[$I, E, A, \rho $][.5][below];
	
	\notation{4}{A}{B1}[1]
	\notation{4}{B1}{B}[2]
	\notation{4}{B}{C}[3] 
	
	\draw[thick, inner color=white] (C) circle (4mm);

\end{tikzpicture}\par
\vspace{15mm}
\noindent
Az előző feladatokhoz hasonló algoritmuson haladtam végig, szintén Maxima programmal.\par
\vspace{5mm}
\noindent
Az alábbi lépésekből állt az algoritmusom: \par
\begin{enumerate}
	\item Adatok bevitele.
	\item Az elemi merevségi mátrix függvény és az elemi tömegmátrix függvény megvalósítása az előző feladatokhoz hasonlóan.
	\item Globális merevségi és tömegmátrix összeállítása az elemi merevségi és tömegmátrixokból az előző feladatban ismertetett módon.
	\item A 4-es pontban lévő tömeg figyelembevétele a mátrixban. Mivel a feladat kiírása szerint a nyomatékot elhanyagolhatom, csak a tömeget kell a mátrixban elhelyezni. Mivel a tömeg a 4-es ponthoz tartozik, így a Globális mátrixban oda kell elhelyezni:
	\[M_G[7,7]=M_G[7,7]+m_0\] 
	\item Mátrixok kondenzálása a szabad szabadsági fokok alapján. A lekötött szabadsági fokok, az előző feladatokhoz képest nem változtak:
	\[V_{1}=0 \qquad \Phi_{1}=0 \qquad V_{2}=0\]
	\newpage
	\item Sajátkörfrekvenciák számítása az alábbi képlet segítségével:
	\[det(K_G-\omega^2 \cdot M_G)=0\]
	Tehát \(det(M_G^{-1} \cdot K_G)\) sajátértékei lesznek $\omega^2$ értékei.
	\item Maxima segítségével az így kapott sajátkörfrekvenciákból számítottam a sajátfrekvenciákat:
	\[f_{1,2,3}^c=\dfrac{\omega_{1,2,3}^c}{2\cdot \pi}\]
	Amivel a sajátfrekvenciák:
	\[f_1^c=0,4462\, [Hz] \qquad f_2^c=4,1848\, [Hz] \qquad f_3^c=11,0737\, [Hz]  \]
	Az így számított eredmények legalább 3 tizedesjegyik megegyeznek a SIKEREZ szoftverrel kapott eredményekkel. A SIKEREZ file-t a Maxima file mellé csatolom "sikerd.skz" néven.
\end{enumerate}\par
\end{document}