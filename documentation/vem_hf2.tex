\documentclass{article}
\usepackage[utf8]{inputenc}
\usepackage[hungarian]{babel}

\usepackage{lipsum}
\usepackage{graphicx} % for the includegrapichs
\usepackage{titling}

\usepackage{fancyhdr} % for the header on the first page
\usepackage{pdfpages} % to include pdf
\usepackage{structuralanalysis} % for the figures
\usepackage{tikz-dimline}
\usepackage{wasysym}



\begin{document}
	
	\begin{titlepage}
		\setlength{\headheight}{20pt}
		\lhead{\includegraphics[height=1.5cm]{logo_mm.png}}
		\rhead{\large{\textbf{Végeselem módszer alapjai}}\\
			BMEGEMMAGMV}
		%\vspace{15cm}
		\title{\huge Kötelező házi feladat 2
		}
		\author{Tar Dániel\\GUTOY7}
		\date{\today}
		\maketitle
		\pagenumbering{gobble}
		\thispagestyle{fancy}
		
		\begin{figure}
			\begin{center}
				\includegraphics[height=2cm]{logo_bme_kicsi.eps}
			\end{center}
		\end{figure}
		
	\end{titlepage}
	\newpage

	%\includepdf{vemalaphf1.pdf}
	 \includepdf[picturecommand*={
	    	\put(460,759){Tar Dániel}
	    	\put(460,742){GUTOY7}
	    	\put(280,680){2}
	    	\put(325,680){1}
	    	\put(368,680){2}
	    	\put(410,680){2}
	    	\put(115,63){eredmeny1}
	    	\put(245,63){eredmeny2}
	    	\put(370,63){eredmeny3}
	    	\put(490,63){eredmeny4}
	 }]{vemalaphf4.pdf}
	\newpage
	
	
	\setlength{\headheight}{0pt}
	\tableofcontents
	\newpage
	
	\pagenumbering{arabic}	
	\setcounter{page}{1}

	
	
	

	
	%\newpage
	
	\section{Feladat}
	
	A házifeladat kód alapján az adatok SI mértékegység alapján:
	
	\begin{table}[h!]
		\begin{center}
			\caption{Adatok}
			\label{tab:table1}
			\begin{tabular}{c|c|c|c|c|c} % <-- Alignments: 1st column left, 2nd middle and 3rd right, with vertical lines in between
				$a$ & $m_{0}$ & $b$ & $d$ & $E$ & $/ro$\\
				$[m]$ & $[kg]$ & $[m]$ & $[m]$ & $[Pa]$ & $[kg/m^{3}]$\\
				\hline
				1.2 & 15 & 6 & 35 & $190\cdot10^3$ & 6500\\
				%2 & 10.1 & b\\
				%3 & 23.113231 & c\\
			\end{tabular}
		\end{center}
	\end{table}
	
	\begin{flushleft}
		A terheléseket arányosan és mindenhol a pozitív irányba vettem fel, hogy megegyezzen a feladatleírásban szereplő ábrával.
	\end{flushleft}
	
	%\newcommand{\newCommandName}{text to insert}
	\newcommand{\sugar}{2}
	\newcommand{\ab}{600}
	\newcommand{\bc}{120}
	\newcommand{\acv}{701}
	\newcommand{\ac}{720}
	
	
	\begin{figure}[h!]
		
		
		\begin{center}
			\begin{tikzpicture}
			\scaling{.015};
			
			% x and y axis
			\draw[->] (0,0)--(12,0) node[right]{$x$};
			\draw[->] (0,0)--(0,2) node[above]{$y$};
			
			% auxiliary points
			\point{a}{0}{0};
			\point{a1}{0}{-\sugar};
			\point{a2}{0}{\sugar};
			
			\point{b}{\ab}{0};
			\point{c}{\ac}{0};
			\point{c1}{\acv}{-\sugar};
			\point{c2}{\acv}{\sugar};
			\point{d}{200}{0};
			
			\draw[thick] (10.8,0) circle (0.3cm);
			
			% stucture
			\beam{2}{a1}{c1};
			\beam{2}{a2}{c2};
			
			
			%\support{type}{insertion point}[rotation];
			\support{2}{b};
			\support{3}{a}[-90];
			
			\notation{1}{a}{$A$}[below=8mm];
			\notation{1}{b}{$B$}[below=10mm];
			\notation{1}{c}{$C$}[below=10mm];
			\notation{1}{d}{$\diameter35~mm$}[above=1mm];
			\notation{1}{c2}{$m_{0}$}[above right=2mm];
			
			% dimensions
			\dimensioning{1}{a}{b}{-1.8}[$6~m$];
			\dimensioning{1}{b}{c}{-1.8}[$1.2~m$];
			%\dimensioning{2}{c1}{c2}{-1}[35];
			%\dimensioning{2}{f}{e}{-1}[$54$];
			
			
			
			
			
			
			\end{tikzpicture}
		\end{center}	
		\caption{Méretarányos ábra és a kényszerek}
	\end{figure}
	
	\newpage
		
	\section{Feladat}
	This formula $f(x) = x^2$ is an example.
	\begin{equation}
	f(x)=x^2
	\end{equation}
	This formula $f(x) = x^2$ is an example.
	\section{Feladat}
	This formula $f(x) = x^2$ is an example.
	
	
	
	

\end{document}
